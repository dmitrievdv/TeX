\documentclass[12pt]{article}
\usepackage[utf8x]{inputenc}
\usepackage[english,russian]{babel}
\usepackage[section]{placeins}
\usepackage{upgreek}
\usepackage{graphicx}
\usepackage[left=2cm, right=1.5cm, top=1.5cm, bottom=1.5cm]{geometry}
\usepackage{caption}
\usepackage{subcaption}
\usepackage{indentfirst}
\usepackage{wrapfig}
\usepackage{amsmath}
\usepackage{gensymb}
\usepackage{setspace}
\usepackage{placeins}
\graphicspath{{images/}}

\begin{document}

\title{Различные системы отсчета}
\maketitle

Начало координат всех систем отсчета помещено в центр звезды.

\begin{itemize}
\item \textbf{Система отсчета наблюдателя}
Система отсчета, в которой ось $Z$ сонаправлена с лучом зрения. Ось вращения звезды отклонена от оси $Z$ (луча зрения) на угол $i$ и лежит в плоскости $XZ$. Угол между осью магнитосферы и осью вращения --- $\alpha$. Угол между плоскостью, содержащей ось вращения и ось магнитосферы, и плоскостью $XZ$ --- $\psi$. 
\item \textbf{Система отсчета звезды}
Повернутая вокруг оси $Y$ система отсчета наблюдателя на угол $i$. Ось вращения в итоге совпадает с осью $Z$. 
\item \textbf{Система отсчета магнитосферы}
Ось магнитосферы в этой системе отсчета совпадает с осью $Z$, а ось вращения лежит в плоскости $XZ$ 
\item \textbf{Рабочая система отсчета}
Луч зрения --- ось $Z$, ось магнитосферы в плоскости $ZY$
\end{itemize}

\section{Переходы между системами отсчета}

\begin{itemize}
\item \textbf{Наблюдатель, звезда}

\begin{equation}\label{eq:obs-to-star}
\begin{aligned}
x_\star &= x_\text{obs} \cos(i) - z_\text{obs}\sin(i), \\
y_\star &= y_\text{obs}, \\
z_\star &= x_\text{obs} \sin(i) + z_\text{obs}\cos(i). \\
\end{aligned}
\end{equation}

\begin{equation}\label{eq:star-to-obs}
\begin{aligned}
x_\text{obs} &= x_\star \cos(i) + z_\star\sin(i), \\
y_\text{obs} &= y_\star, \\
z_\text{obs} &= -x_\star \sin(i) + z_\star\cos(i). \\
\end{aligned}
\end{equation}

\item \textbf{Магнитосфера, звезда}

\begin{equation}\label{eq:mag-to-star}
\begin{aligned}
x_\star &= x_\text{mag} \cos(\psi)\cos(\alpha) + y_\text{mag}\sin(\psi) - z_\text{mag}\cos(\psi)\sin(\alpha), \\
y_\star &= - x_\text{mag}\sin(\psi)\cos(\alpha) + y_\text{mag}\cos(\psi) + z_\text{mag}\sin(\psi)\sin(\alpha),  \\
z_\star &= x_\text{mag} \sin(\alpha) + z_\text{mag}\cos(\alpha). \\
\end{aligned}
\end{equation}

\begin{equation}\label{eq:star-to-mag}
\begin{aligned}
x_\text{mag} &= x_\star \cos(\psi)\cos(\alpha) - y_\star\sin(\psi)\cos(\alpha) + z_\star\sin(\alpha), \\
y_\text{mag} &= x_\star\sin(\psi) + y_\star\cos(\psi),  \\
z_\text{mag} &= -x_\star\cos(\psi)\sin(\alpha) + y_\star\sin(\psi)\sin(\alpha) + z_\star\cos(\alpha). \\
\end{aligned}
\end{equation}

\item\textbf{Магнитосфера, наблюдатель}

Подставляя \eqref{eq:obs-to-star} в \eqref{eq:star-to-mag}, получим:
%
\begin{equation}\label{eq:obs-to-mag}
\begin{aligned}
x_\text{mag} &= x_\text{obs}(\cos(i)\cos(\alpha)\cos(\psi) - \sin(i)\sin(\alpha)) - y_\text{obs}\cos(\alpha)\sin(\psi) \\
             &+ z_\text{obs}(\cos(i)\sin(\alpha) - \sin(i)\cos(\alpha)\cos(\psi)), \\
y_\text{mag} &= x_\text{obs}\cos(i)\sin(\psi) + y_\text{obs}\cos(\psi) - z_\text{obs}\sin(i)\sin(\psi),  \\
z_\text{mag} &= x_\text{obs}(\sin(i)\cos(\alpha) - \cos(i)\cos(\psi)\sin(\alpha))  + y_\text{obs}\sin(\psi)\sin(\alpha) \\
             &+ z_\text{obs}(\cos(i))\cos(\alpha) + \sin(i)\cos(\psi)\sin(\alpha)). \\
\end{aligned}
\end{equation}

А \eqref{eq:mag-to-star} в \eqref{eq:star-to-obs}:

\begin{equation}\label{eq:mag-to-obs}
\begin{aligned}
x_\text{obs} &= x_\text{mag}(\cos(\psi)\cos(\alpha)\cos(i) + \sin(\alpha)\sin(i)) + y_\text{mag}\sin(\psi)\cos(i) \\
             &+ z_\text{mag}(\cos(\alpha)\sin(i) - \cos(\psi)\sin(\alpha)\cos(i)), \\
y_\text{obs} &= -x_\text{mag}\sin(\psi)\cos(\alpha) + y_\text{mag}\cos(\psi) + z_\text{mag}\sin(\psi)\sin(\alpha), \\
z_\text{obs} &= x_\text{mag}(\sin(\alpha)\cos(i) - \cos(\psi)\cos(\alpha)\sin(i)) + y_\text{mag}\sin(\psi)\sin(i) \\
             &+ z_\text{mag}(\cos(\alpha)\cos(i) + \cos(\psi)\sin(\alpha)\sin(i)), \\
\end{aligned}
\end{equation}

\item \textbf{Рабочая, наблюдатель}

Чтобы перейти в \textbf{рабочую систему отчета} нужно повернуть \textbf{систему наблюдателя} на угол $\varepsilon$ вокруг оси $Z$, определяемый соотношениями \eqref{eq:mag-to-obs}, если подставить в них вектор оси магнитосферы $\left\lbrace 0,\ 0,\ 1 \right\rbrace$:
%
\begin{equation}\label{eq:work-orientation}
\begin{aligned}
\sin(\varepsilon) &= \frac{(\cos(\alpha)\sin(i) - \cos(\psi)\sin(\alpha)\cos(i))}{\sqrt{1-\cos^2(\theta)}}, \\
\cos(\varepsilon) &= \frac{\sin(\psi)\sin(\alpha)}{\sqrt{1-\cos^2(\theta)}}, \\
\cos(\theta) &= \cos(\alpha)\cos(i) + \cos(\psi)\sin(\alpha)\sin(i), \\
\end{aligned}
\end{equation}
%
здесь $\theta$ --- угол наклона оси магнитосферы к лучу зрения. Тогда:
%
\begin{equation}\label{eq:obs-to-work}
\begin{aligned}
x_\text{work} &= x_\text{obs}\cos(\varepsilon) - y_\text{obs}\sin(\varepsilon), \\
y_\text{work} &= x_\text{obs}\sin(\varepsilon) + y_\text{obs}\cos(\varepsilon) , \\
z_\text{work} &= z_\text{obs}, \\
\end{aligned}
\end{equation}
%
\begin{equation}\label{eq:work-to-obs}
\begin{aligned}
x_\text{obs} &= x_\text{work}\cos(\varepsilon) + y_\text{work}\sin(\varepsilon), \\
y_\text{obs} &= - x_\text{work}\sin(\varepsilon) + y_\text{work}\cos(\varepsilon) , \\
z_\text{obs} &= z_\text{work}, \\
\end{aligned}
\end{equation}

Остальные переходы можно получить перемножением матриц.
\end{itemize}
\end{document}