\documentclass[12pt]{article} 
\usepackage[utf8x]{inputenc}
\usepackage[english,russian]{babel}
\usepackage{cyr}
\usepackage{epsf}
\usepackage{pazh}
\tightenlines

\voffset=10mm 
\hoffset=0mm
\parindent 10mm

%################## some definitions ###########################
\def\phflux{фот$\cdot$см$^{-2}$с$^{-1}$кэВ$^{-1}$}
\def\phfl{фот$\cdot$см$^{-2}$с$^{-1}$}
\def\ergs{эрг~с$^{-1}$}
\def\*{$^{*}$}
\def\а{$^{\mbox{\small а}}$}
\def\б{$^{\mbox{\small б}}$}
\def\в{$^{\mbox{\small в}}$}
\def\г{$^{\mbox{\small г}}$}
\def\д{$^{\mbox{\small д}}$}
\def\ергс{эрг~с$^{-1}$}
\def\ергсм{эрг~см$^{-2}$~с$^{-1}$}
\def\etal{{et~al.}}
%################################################################
\sloppypar

\begin{document}
\baselineskip 21pt

УДК 524.354.4

\title{\bf ВСПЫШКА РЕНТГЕНОВСКОГО ТРАНЗИЕНТА ...}

\author{\bf \hspace{-1.3cm}\copyright\, 2006 г. \ \ 
А.А.Иванов\affilmark{1*}, В.В.Сидоров\affilmark{1,2}}

\affil{
{\it Ин-т космических исследований РАН, Москва}$^1$\\ 
{\it Ин-т астрофизики Общества им. Макса Планка, 
Гархинг, Германия}$^2$}

\vspace{2mm}
%\received{~~~~~~~~}
\received{\today}
%\revised{}

\sloppypar 
\vspace{2mm}
\noindent
В ходе наблюдения .... 

\noindent
{\bf Ключевые слова:\/} рентгеновские источники, транзиенты, аккреция

\noindent
{\bf PACS codes:\/} ?????

\vfill
\noindent\rule{8cm}{1pt}\\
{$^*$ Электронный адрес $<$ivanov@hea.iki.rssi.ru$>$}

\clearpage

%***************************************************************
\section*{ВВЕДЕНИЕ}
\noindent
Источник ... 


%*************************************************************
\section*{НАБЛЮДЕНИЯ}
\noindent
Международная обсерватория гамма-лучей INTEGRAL (Винклер и
др., 2003) была выведена ... 

%*************************************************************
\section*{РЕЗУЛЬТАТЫ}
\noindent
Впервые ....

\subsection*{ВРЕМЕННОЙ ПРОФИЛЬ}
\noindent
Для выяснения ....

\subsection*{СПЕКТР ИЗЛУЧЕНИЯ}
\noindent

%*************************************************************
\section*{ОБСУЖДЕНИЕ}
\noindent
Транзиенты, подобные ....
 


Работа основана на наблюдательных данных, полученных ...
Исследование выполнено при финансовой поддержке Российского
фонда фундаментальных исследований (грант ????) и программы
Президента РФ поддержки научных школ (грант НШ-???).
 
\pagebreak   
%****************************************************************
 
\begin{references}

\reference{\it 
Винклер и др.} (C. Winkler, T.J.-L. Courvoisier, G. Di Cocco 
\etal), \aap\ {\bf 411}, L1 (2003). 

\reference{\it 
Льюин и др.} (W.H.G. Lewin, J. van Paradijs, R.E. Taam),
\ssr\ {\bf 62}, 223 (1993).

\reference{\it 
Ревнивцев М.Г., Сюняев Р.А., Варшалович Д.А. и др.\/}, \pazh\
{\bf 30}, 430 (2004). 

\reference {\it 
Сюняев, Титарчук} (R.A. Sunyaev, L.G. Titarchuk) \dd\ \aap\ 
{\bf 86}, 121 (1980).

\end{references}

\clearpage
%---------------------------------------------------------------
\begin{table}[t]

\vspace{6mm}
\centering
{{\bf Таблица 1.} Результаты аппроксимации ...}\label{meansp} 

\vspace{5mm}\begin{tabular}{l|c|c|c|c|c} \hline\hline
{Модель}&$kT,$&$\alpha$б\ &$kT_{\rm bb},$в\ &$L_{\rm bb},$в\ &$\chi^2(N)$г\ \\ 
        & кэВ&            & кэВ         &$10^{38} \ \mbox{эрг с}^{-1}$&\\ \hline
PL   &              &$2.75\pm0.08$&            &           &1.02 (22)\\
WS   &$ 49.9\pm5.7 $&             &            &           &1.07 (22)\\
ST   &$ 27.9\pm2.1 $&$2.51\pm0.10$&            &           &1.02 (21)\\
TB   &$ 28.8\pm1.9 $&             &            &           &1.37 (22)\\
TB+BB&$ 36.3\pm3.5 $&             &$1.7\pm0.7$ &$1.6\pm0.4$&0.75 (20)\\
ST+BB&$ 16.8\pm3.7 $&$2.08\pm0.57$&$1.7$д\    &$1.4\pm0.5$&0.78 (20)\\ \hline
\multicolumn{6}{l}{}\\ [-3mm]
\multicolumn{6}{l}{а обозначения моделей даны в тексте}\\
\multicolumn{6}{l}{б фотонный индекс}\\
\multicolumn{6}{l}{в параметры мягкой компоненты излучения
  (чернотельная}\\
\multicolumn{6}{l}{\ \ \  температура и болометрическая
  светимость для $d=8$ кпк)}\\
\multicolumn{6}{l}{г значение $\chi^2$ наилучшей
  аппроксимации, нормированное на $N$}\\
\multicolumn{6}{l}{\ \ \ ($N$ -- число степеней
  свободы)}\\
\multicolumn{6}{l}{д зафиксированный параметр} \\
\end{tabular}
\end{table}
%---------------------------------------------------------------
\clearpage

%*************************************************************** 
\centerline {\bf ПОДПИСИ К РИСУНКАМ}
\vspace{1 cm} 

Рис.1.~Кривая блеска ...

Рис.2.~Рентгеновское изображение ...

\clearpage
%xxxxxxxxxxxxxxxxxxxxxxxxxxxxxxxxxxxxxxxxxxxxxxxxxxxxxxxxxxxxxx
\begin{figure}[h]
\epsfxsize=19cm
\hspace{-2cm}%\epsffile{fig1.ps}

\caption{\rm }
%\caption{\rm Кривая блеска ...}
\end{figure}

%xxxxxxxxxxxxxxxxxxxxxxxxxxxxxxxxxxxxxxxxxxxxxxxxxxxxxxxxxxxxx
%---------------------------------------------------------------
\end{document}
