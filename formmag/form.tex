\documentclass{article}     
\usepackage[utf8x]{inputenc}
\usepackage[english,russian]{babel}
\usepackage[section]{placeins}
\usepackage{upgreek}
\usepackage{graphicx}
\usepackage{caption}
\usepackage{subcaption}
\usepackage{wrapfig}
\usepackage{amsmath}
\graphicspath{{images/}}

\textheight 24.5cm % 29.7-2-2=25.7
\textwidth 17cm % 21-2.5-1.5=17.0
\hoffset -0.04cm %2.5-2.54=-0.04 слева 3см
\voffset -0.54cm %2-2.54=0.54 сверху 2см
\oddsidemargin 0cm
\headheight 0cm
\headsep 0cm
\topmargin 0cm

\begin{document}

\title{Некоторые формулы для магнитосферы}
\maketitle

Расстояние по линии тока от полярного угла и расстояния от звезды

\[D(\theta) = \frac{r_m}{2}\left( \ln(1+\sqrt{2}) + \sqrt{2} - \ln(\cos\theta + \sqrt{1+\cos^2\theta}) - \cos\theta\sqrt{1+\cos^2\theta} \right)\]

\end{document}