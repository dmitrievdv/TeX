\documentclass{article}     
\usepackage[utf8]{inputenc}
\usepackage[russian]{babel}
\begin{document}

Рассмотрим задачу расчета профиля эмиссионной линии $I_{\nu}$. Можно записать

$$I_{\nu} = \int \limits_{S} I_{xy}(\nu) dS$$

Где $S$ - площадь проекции области, в которой происходит излучение, на плоскость $XY$ (ось $z$ направлена на наблюдателя) 

Рассмотрим $I_{xy}$

$$I_{xy}(\nu) =  \int \limits_{z_0}^{z_k} S(z)k(\nu, z)e^{-\tau}dz$$
$$\tau = \int \limits_z^{z_k} k(\nu)dz$$
$$S(z) = \frac{2h\nu_0^3}{c^2}\left(\frac{n_l(z)}{n_u(z)} \frac{g_u}{g_l} - 1 \right)^{-1} $$
$$k(\nu, z) = 0,02654f_{lu}\alpha(\nu, z)n_l$$
$$\alpha(\nu, z) = \frac{1}{\sqrt{\pi} \Delta\nu_D} \exp\left( -\left[ \frac{\nu}{\Delta\nu_D} - \frac{v_z(z)}{u}\right]^2\right)$$

Где $S(z)$ - функция источника, $k(\nu, z)$ - коэффициент поглощения, $\nu_0$ - центральная частота линии, $u$ - тепловая скорость, $\Delta\nu_D$ - доплеровская ширина линии, $v_z(z)$ - лучевая скорость вещества в точке $(x, y, z)$. Разобьем промежуток от $z_0$ до $z_k$ (в общем случае это может быть не один промежуток, а несвязная совокупность промежутков $[z_0,~z_1),[z_1,~z_2)...[z_{k-1},~z_k)$) на $n$ непересекающихся промежутков $\triangle z_i$

$$\triangle z_i = \left( z_i^0, z_i^k \right] $$
$$\triangle z_i > \triangle z_{i-1}$$
$$ z_i^o = z_i - \frac{\Delta z_i}{2},\ z_i^k = z_i + \frac{\Delta z_i}{2}$$

$S(z)$ и $k(z)$ представим как сумму простых функций, принимающих значения $S_i$, $k_i$ на всем промежутке $\triangle z_i$, а в остальных точках равных нулю

$$S(z) = \sum \limits_i \mathbf{1}_{\triangle z_i}(z) S_i$$ 
$$k(\nu, z) = \sum \limits_i \mathbf{1}_{\triangle z_i}(z) k_i(\nu)$$
$$S_i = S(z_i)$$ 
$$k_i = k(\nu, z_i)$$

Где $\mathbf{1}_{\triangle z_i}(z)$ - функция принадлежности $z$ промежутку $\triangle z_i$. В дальнейшем будем рассматривать рассчет интенсивности излучения на заданной частоте $\nu$, и зависимость функций $I_{xy}$, $k_i$ от $\nu$ явно указывать не будем.

Тогда мы можем разбить интеграл $I_{xy}$ на сумму интегралов по $\triangle z_i$

$$I_{xy} = \sum \limits_i \int \limits_{\triangle z_i} S_i k_i e^{-\tau}dz = \sum \limits_i S_i k_i \int \limits_{\triangle z_i} e^{-\tau}dz$$

Рассмотрим отдельно $e^{-\tau}$

$$e^{-\tau} = \exp \left( -\int \limits_z^{z_k} kdz\right) = 
\exp \left( -\int \limits_z^{z_i^k} k_idz - \int \limits_{z_{i+1}^o}^{z_k} kdz\right) = \exp \left( -\int \limits_z^{z_i^k} k_idz\right) e^{ -\tau_i}  $$

Где 
$$\tau_i = \int \limits_{z_{i+1}^o}^{z_k} kdz = \sum \limits_{j=i+1}^n k_j\Delta z_j$$

Есть число, независящее от $z$. Следовательно 

$$I_{xy} = \sum \limits_i S_i k_i e^{-\tau_i} \int \limits_{\triangle z_i} \exp\left( -\int \limits_z^{z_i^k}k_idz\right)dz = \sum \limits_i S_i k_i e^{-\tau_i} \int \limits_{z_i^o}^{z_i^k} \exp\left( k_i\left(z - z_i - \frac{\Delta z_i}{2}\right)\right)dz = $$
$$ =\sum \limits_i S_i k_i e^{-\tau_i} \frac{1}{k_i} \exp\left( k_i\left(z - z_i - \frac{\Delta z_i}{2}\right)\right)\Biggr|_{z_i - \frac{\Delta z_i}{2}}^{z_i + \frac{\Delta z_i}{2}} =  \sum \limits_i S_i e^{-\tau_i}\left( 1 - e^{-k_i\Delta z_i} \right) $$

В итоге имеем

$$ I_{xy}(\nu) = \sum \limits_i S_i e^{-\tau_i(\nu)}\left( 1 - e^{-k_i(\nu)\Delta z_i} \right) $$
$$\tau_i = \sum \limits_{j=i+1}^n k_j(\nu)\Delta z_j$$



\end{document}